\documentclass[11pt]{article}

% Font & ngôn ngữ tiếng Việt (pdfLaTeX)
\usepackage[utf8]{inputenc}
\usepackage[T5]{fontenc}


% Biblatex + biber
\usepackage[backend=biber, style=numeric-comp, autolang=other]{biblatex}

%\addbibresource{references.bib} # tham khao

% Toán học & font Times
\usepackage{amsmath, amssymb, amsfonts, bm}

% Bảng biểu & căn lề
\usepackage{longtable}
\usepackage{array}
\usepackage{booktabs}

% Đồ họa & màu sắc
\usepackage{graphicx}
\usepackage{xcolor}
\usepackage{float}
\usepackage{subcaption}

% Liên kết & tham chiếu
\usepackage{hyperref}
\hypersetup{
    colorlinks=true,
    linkcolor=blue,
    urlcolor=red,
    pdftitle={Overleaf Example},
    pdfpagemode=FullScreen,
}

% Dấu tick và x
\usepackage{pifont}
\newcommand{\xmark}{\ding{55}}
\newcommand{\cmark}{\ding{51}}

% Tiêu đề tùy chỉnh
\usepackage{titling}
\setlength{\droptitle}{-10em}
\renewcommand{\maketitle}{
    \begin{center}
        \fontsize{18}{20}\selectfont\textbf{Avoid Training \& Predicting Data Conflict using Feast}\\[1em]
        \fontsize{14}{16}\selectfont Nhóm TimeSeries - GRID070\\[0.5em]
        \fontsize{14}{16}\selectfont Ngày 9 tháng 10 năm 2025
    \end{center}
    \vspace{1.5em}
}

% Format section (không đánh số)
\usepackage{titlesec}
\titleformat{\section}{\normalfont\Large\bfseries}{}{0em}{}

% Code block
\usepackage{listings}
\definecolor{backcolour}{rgb}{0.95,0.95,0.92}
\lstset{
    backgroundcolor=\color{backcolour},
    basicstyle=\ttfamily\footnotesize,
    breaklines=true,
    numbers=left,
    numberstyle=\tiny\color{gray},
    captionpos=b,
    language=python, % Đặt ngôn ngữ mặc định là bash cho lstlisting
    morekeywords={python, dvc, git, conda, mkdir, cd, pip, mklink, del}, % Thêm keywords
    commentstyle=\color{gray}, % Màu cho comment
    keywordstyle=\color{blue},
    stringstyle=\color{red}
}

% Hộp màu
\usepackage[many]{tcolorbox}
\definecolor{lightgreenbox}{rgb}{0.85,0.95,0.85}
\newtcolorbox{summarybox}{
    colback=lightgreenbox,
    colframe=green!50!black,
    boxsep=5pt, arc=4pt,
    boxrule=0.5pt,
    left=10pt, right=10pt,
    top=10pt, bottom=10pt,
}


% Layout trang
\setlength{\topmargin}{-0.7in}
\setlength{\textheight}{9.25in}
\setlength{\oddsidemargin}{0in}
\setlength{\textwidth}{6.8in}

%%%%%%%%%%%%%%%%%%%%%%%%%%%%%%%%%%%%%%%%%%%%%%%%%%%%%%%%%%%%%%%%%%%%%%%%%%%%%
\begin{document}

\maketitle

\begin{summarybox}
Nội dung về được chia thành 5 phần chính:
    \begin{itemize}
	\item \textbf{Revision: Data Versioning và ETL pipeline}	
	\item \textbf{Overview: Giới thiệu vòng đời của 1 dự án Học Máy}
	\item \textbf{Feature Store và Feast là gì ?}
	\item \textbf{Case Study}
	\item \textbf{Final Revision: ôn lại mọi khái niệm trong bài cùng team Time Series}
    \end{itemize}
\end{summarybox}

% --- PHẦN 1 ---

\section{Intro: Nhắc lại Data Versoning và quy trình ETL}	

\section{Giới thiệu vòng đời của 1 dự án Học Máy}

Vòng đời dự án ML -> Cơ sở hạ tầng AI.

6 vai trò trong Team AI. 

Vấn đề chính trong ML Proj life cycle. 

\section{Feature Store và Feast là gì ?}
Vấn đề Feature Store giải quyết -> Các thành phần chính trong Feature Store pipeline -> Lợi ích của Feature Store trong bố cảnh X. (liệt kê các lợi ích)

Feast thật sự là gì trong Feature  ? (triết lý của Feast) 	-> Feast Pipeline  -> Các thành phần của Feast 



\section{Case Study}



\section{Final Revision: ôn lại mọi khái niệm trong bài cùng team Time Series}


%\printbibliography % Bỏ comment dòng này nếu bạn có tệp .bib

\end{document}