\documentclass[11pt]{article}

% Font & ngôn ngữ tiếng Việt (pdfLaTeX)
\usepackage[utf8]{inputenc}
\usepackage[T5]{fontenc}


% Biblatex + biber
\usepackage[backend=biber, style=numeric-comp, autolang=other]{biblatex}

%\addbibresource{references.bib} # tham khao

% Toán học & font Times
\usepackage{amsmath, amssymb, amsfonts, bm}

% Bảng biểu & căn lề
\usepackage{longtable}
\usepackage{array}
\usepackage{booktabs}

% Đồ họa & màu sắc
\usepackage{graphicx}
\usepackage{xcolor}
\usepackage{float}
\usepackage{subcaption}

% Liên kết & tham chiếu
\usepackage{hyperref}
\hypersetup{
    colorlinks=true,
    linkcolor=blue,
    urlcolor=red,
    pdftitle={Overleaf Example},
    pdfpagemode=FullScreen,
}

% Dấu tick và x
\usepackage{pifont}
\newcommand{\xmark}{\ding{55}}
\newcommand{\cmark}{\ding{51}}

% Tiêu đề tùy chỉnh
\usepackage{titling}
\setlength{\droptitle}{-10em}
\renewcommand{\maketitle}{
    \begin{center}
        \fontsize{18}{20}\selectfont\textbf{Avoid Training \& Predicting Data Conflict using Feast}\\[1em]
        \fontsize{14}{16}\selectfont Nhóm TimeSeries - GRID070\\[0.5em]
        \fontsize{14}{16}\selectfont Ngày 9 tháng 10 năm 2025
    \end{center}
    \vspace{1.5em}
}

% Format section (không đánh số)
\usepackage{titlesec}
\titleformat{\section}{\normalfont\Large\bfseries}{}{0em}{}

% Code block
\usepackage{listings}
\definecolor{backcolour}{rgb}{0.95,0.95,0.92}
\lstset{
    backgroundcolor=\color{backcolour},
    basicstyle=\ttfamily\footnotesize,
    breaklines=true,
    numbers=left,
    numberstyle=\tiny\color{gray},
    captionpos=b,
    language=python, % Đặt ngôn ngữ mặc định là bash cho lstlisting
    morekeywords={python, dvc, git, conda, mkdir, cd, pip, mklink, del}, % Thêm keywords
    commentstyle=\color{gray}, % Màu cho comment
    keywordstyle=\color{blue},
    stringstyle=\color{red}
}

% Hộp màu
\usepackage[many]{tcolorbox}
\definecolor{lightgreenbox}{rgb}{0.85,0.95,0.85}
\newtcolorbox{summarybox}{
    colback=lightgreenbox,
    colframe=green!50!black,
    boxsep=5pt, arc=4pt,
    boxrule=0.5pt,
    left=10pt, right=10pt,
    top=10pt, bottom=10pt,
}


% Layout trang
\setlength{\topmargin}{-0.7in}
\setlength{\textheight}{9.25in}
\setlength{\oddsidemargin}{0in}
\setlength{\textwidth}{6.8in}

%%%%%%%%%%%%%%%%%%%%%%%%%%%%%%%%%%%%%%%%%%%%%%%%%%%%%%%%%%%%%%%%%%%%%%%%%%%%%
\begin{document}

\maketitle

\begin{summarybox}
Nội dung về được chia thành 5 phần chính:
    \begin{itemize}
	\item \textbf{Revision: Data Versioning và ETL pipeline}
	\item \textbf{Overview: Giới thiệu vòng đời của 1 dự án Học Máy}
	\item \textbf{Feature Store và Feast là gì ?}
	\item \textbf{Case Study}
	\item \textbf{Final Revision: ôn lại mọi khái niệm trong bài cùng team Time Series}
    \end{itemize}
\end{summarybox}

% --- PHẦN 1 ---
\section{Intro: Từ ETL, Processing, Data Versoning đến Feature Store}
Hành trình của dữ liệu bắt đầu từ quy trình ETL (Extract, Transform, Load) để thu thập, chuẩn hóa dữ liệu thô từ nhiều nguồn đa dạng và làm sạch, biến đổi chúng rồi lưu trữ vào 1 trung tâm lưu trữ tập trung như Data Warehouse. \\

Tuy nhiên, thực tế có dữ liệu "động" cập nhập liên tục (Streaming data như tin nhắn, đơn hàng) và dữ liệu "tĩnh" được lưu ít thay đổi (static như sách, CCCD), tạo ra nhu cầu cho các phương pháp xử lý tinh vi hơn như Apache Spark cho phép ta xử lý đa luồng dữ liệu phức tạp cả tĩnh và động. Đồng thời hỗ trợ các tác vụ truy vấn (SparkSQL) và học máy (MLlib).

Chính sự


\subsection{ML/AI Project Lifecycle - Giới thiệu vòng đời của 1 dự án Học Máy}
\textbf{Data Engineering}
\textbf{Modeling}
\textbf{Deployment}
\textbf{Business analysis}
\textbf{AI Infrastructure - Cơ sở hạ tầng của 1 dự án AI}
\textbf{6 vai trò trong nhóm phát triển AI}

\section{Feature Store và Feast là gì ?}
Vấn đề chính trong ML Project life cycle.
Vấn đề Feature Store giải quyết
-> Các thành phần chính trong Feature Store pipeline
-> Lợi ích của Feature Store trong bố cảnh X. (liệt kê các lợi ích).

Feast thật sự là gì trong Feature ? (triết lý của Feast)
-> Feast Pipeline -> Các thành phần của Feast.

\section{Case Study}



\section{Final Revision: ôn lại mọi khái niệm trong bài cùng team Time Series}


%\printbibliography % Bỏ comment dòng này nếu bạn có tệp .bib

\end{document}