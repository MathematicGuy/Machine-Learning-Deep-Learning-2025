\documentclass[11pt]{article}
% Font & ngôn ngữ tiếng Việt (pdfLaTeX)
\usepackage[utf8]{inputenc}
\usepackage[T5]{fontenc}


% Biblatex + biber
\usepackage[backend=biber, style=alphabetic, sorting=ynt]{biblatex}
\addbibresource{references.bib}
\usepackage{listings}
\usepackage{xcolor}
\lstdefinestyle{shell}{
	language=bash,
	basicstyle=\ttfamily\small,
	commentstyle=\color{green!60!black},
	keywordstyle=\color{blue},
	stringstyle=\color{red},
	emphstyle=\color{purple},
	backgroundcolor=\color{gray!10},
	showstringspaces=false,
	frame=single,
	breaklines=true,
	emph={[1]\$\ },  % This is the key part: highlight the prompt
	emph=[1]{git, conda, mkdir, cd, pip} % Also emphasize key commands
}

% Toán học & font Times
\usepackage{amsmath, amssymb, amsfonts, bm}

% Bảng biểu & căn lề
\usepackage{longtable}
\usepackage{array}
\usepackage{booktabs}

% Đồ họa & màu sắc
\usepackage{graphicx}
\usepackage{xcolor}
\usepackage{float}
\usepackage{subcaption}

% Liên kết & tham chiếu
\usepackage{hyperref}
\hypersetup{
    colorlinks=true,
    linkcolor=blue,
    urlcolor=red,
    pdftitle={Overleaf Example},
    pdfpagemode=FullScreen,
}
\usepackage{bookmark}

% Dấu tick và x
\usepackage{pifont}
\newcommand{\xmark}{\ding{55}}
\newcommand{\cmark}{\ding{51}}

% Tiêu đề tùy chỉnh
\usepackage{titling}
\setlength{\droptitle}{-10em}
\renewcommand{\maketitle}{%
    \begin{center}
        \fontsize{18}{20}\selectfont\textbf{Data Version Control (DVC) \\[0.2em] in Machine Learning Projects}\\[1em]
        \fontsize{14}{16}\selectfont Nhóm MLOps\\[0.5em]
        \fontsize{14}{16}\selectfont Ngày 18 tháng 10 năm 2025
    \end{center}
    \vspace{1.5em}
}

% Format section (không đánh số)
\usepackage{titlesec}
\titleformat{\section}{\normalfont\Large\bfseries}{}{0em}{}

% Code block
\usepackage{listings}
\definecolor{backcolour}{rgb}{0.95,0.95,0.92}
\lstset{
    backgroundcolor=\color{backcolour},
    basicstyle=\ttfamily\footnotesize,
    breaklines=true,
    numbers=left,
    numberstyle=\tiny\color{gray},
    captionpos=b
}

% Hộp màu
\usepackage[many]{tcolorbox}
\definecolor{lightgreenbox}{rgb}{0.85,0.95,0.85}
\newtcolorbox{summarybox}{
    colback=lightgreenbox,
    colframe=green!50!black,
    boxsep=5pt, arc=4pt,
    boxrule=0.5pt,
    left=10pt, right=10pt,
    top=10pt, bottom=10pt,
}


% Layout trang
\setlength{\topmargin}{-0.7in}
\setlength{\textheight}{9.25in}
\setlength{\oddsidemargin}{0in}
\setlength{\textwidth}{6.8in}

%%%%%%%%%%%%%%%%%%%%%%%%%%%%%%%%%%%%%%%%%%%%%%%%%%%%%%%%%%%%%%%%%%%%%%%%%%%%%
\begin{document}

\maketitle

\begin{summarybox}
Nội dung về được chia thành 2 phần chính, phần 1 giải thích Logistic Regression đến Linear Regression 1 cách tổng quát và trực quan. Phần 2 tập trung giải thích sâu hơn về việc tại sao lại sử dụng những công thức này và ý nghĩa của chúng.
    \begin{itemize}
	\item \textbf{Phần 1: Từ Logistic Regression đến Linear Regression đến Logistic Regression}
	\item \textbf{Phần 2: Mở Rộng: Nhân ma trận với Logistic Regression}
	\item \textbf{Phần 3: Tại Sao}
    \end{itemize}
\end{summarybox}

% --- PHẦN 1 ---
\section{Phần 1: Từ Logistic Regression đến Linear Regression đến Logistic Regression}
\subsection{Phần 1.1: Tổng quan về Linear Regression}
- giải thích thuật ngữ 
+ Linear 
+ Regression

- giải thích công thức qua 1 ví dụ từ đầu đến cuối
- Batch Gradient Descent  và Stochastic Gradient Descent cùng với ưu điểm và nhược điểm
+ Batch is for calc large dataset (calc by batch)
+ Stochastic (shuffle sample - calc)
- code lại sử dụng ví dụ.

\subsection{Phần 1.2: Logistic Regression}
- Liên kết với Linear Regression và vấn đề Logistic Regression giải quyết
- giải thích thuật ngữ
- giải thích công thức qua 1 ví dụ từ đầu đến cuối
- giải thích ý nghĩa công thức
- code lại sử dụng ví dụ.

\section{Phần 2: Mở Rộng: Nhân ma trận với Logisitc Regression}
- Giải thích 1 số lý thuyết đại số tuyến tính trong nhân ma trận. 
+ Nhân nghịch đảo
- Ví dụ với ma trận X và w. Nhân ma trận.
-> tăng tính tương tác. Gợi ý thay biến để ng đọc học qua tương tác.

\section{Phần 3: Tại sao}
\subsection{Tại sao lại nhân nghịch đảo ?}
\subsection{Tại sao lại sử dụng hàm tăng trưởng Sigmoid ?}
\subsection{Tại sao lại sử dụng ln() thay vì log()}
\subsection{Mối liên hệ giữa ln() và Sigmoid ?}
\subsection{Tại sao lại sử dụng loglikelihood thay vì probability}
\subsection{Convex là gì ?}
\subsection{Convex trong Logistic Regression cho tính toán ma trận}


\printbibliography % Bỏ comment dòng này nếu bạn có tệp .bib

\end{document}
