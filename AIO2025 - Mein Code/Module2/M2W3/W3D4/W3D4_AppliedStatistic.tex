\documentclass[11pt]{article}
% Font tiếng việt
\usepackage{fontspec}
\usepackage[vietnamese]{babel}

% Sử dụng bảng dài có thể ngắt trang, bạn cần thêm 2 gói này vào preamble
\usepackage{longtable}
\usepackage{ragged2e}


\usepackage{graphicx}
\usepackage{amsmath, amssymb, amsfonts, bm}
\usepackage{xcolor}
\usepackage{hyperref}
\usepackage{pifont}
\newcommand{\xmark}{\ding{55}}
\newcommand{\cmark}{\ding{51}}
\usepackage{array}
\usepackage{float}

\definecolor{codehighlight}{rgb}{0.95,0.95,0.92} % Using your backcolour


\hypersetup{
    colorlinks=true,
    linkcolor=blue,
    filecolor=magenta,
    urlcolor=red,
    pdftitle={Overleaf Example},
    pdfpagemode=FullScreen,
}

\setmainfont{Times New Roman}
% \setsansfont{Arial} % Not used in the final layout, can comment out
% \setmonofont{Courier New} % Not used in the final layout, can comment out


% Page layout
\setlength{\topmargin}{-.5in}
\setlength{\textheight}{9.25in}
\setlength{\oddsidemargin}{0in}
\setlength{\textwidth}{6.8in}

% No explicit title formatting needed as we'll place it manually
\usepackage{titling}
\setlength{\droptitle}{-10em} % Keep this to prevent default title spacing issues if using \maketitle later for other purposes

\renewcommand{\maketitle}{%
    \begin{center}
        \fontsize{18}{20}\selectfont\textbf{Tuần 1 - Tổng hợp kiến thức Buổi học số 1 và 2}\\[1em]
        \fontsize{14}{16}\selectfont Time-Series Team\\[0.5em]
        \fontsize{14}{16}\selectfont Ngày 17 tháng 7 năm 2025
    \end{center}
    \vspace{1.5em} % Add some space after the custom title block
}


% Remove fancy header/footer as it's not in the image
% \usepackage{fancyhdr}
% \pagestyle{fancy}
% \fancyhf{}
% \renewcommand{\footrulewidth}{0.4pt}
% \lhead{\bfseries AI VIETNAM}
% \rhead{\bfseries aivietnam.edu.vn}
% \fancyfoot[C]{\thepage}

% Section format (không đánh số section) - Keep this
\usepackage{titlesec}
\titleformat{\section}
{\normalfont\Large\bfseries}
{}{0em}{}

% Listings (code block) - Keep this if you plan to use code blocks
\usepackage{listings}
\definecolor{codegreen}{rgb}{0,0.6,0}
\definecolor{codegray}{rgb}{0.5,0.5,0.5}
\definecolor{codepurple}{rgb}{0.58,0,0.82}
\definecolor{backcolour}{rgb}{0.95,0.95,0.92}
\lstdefinestyle{mystyle}{
    backgroundcolor=\color{backcolour},
    commentstyle=\color{codegreen},
    keywordstyle=\color{magenta},
    numberstyle=\tiny\color{codegray},
    stringstyle=\color{codepurple},
    basicstyle=\ttfamily\footnotesize,
    breaklines=true,
    captionpos=b,
    keepspaces=true,
    numbers=left,
    numbersep=5pt,
    tabsize=2,
    showspaces=false,
    showstringspaces=false,
    showtabs=false
}
\lstset{style=mystyle}


% Colored boxes
\usepackage[many]{tcolorbox}
\definecolor{lightgreenbox}{rgb}{0.85,0.95,0.85} % A light green color closer to the image
\newtcolorbox{summarybox}{
    colback = lightgreenbox,
    colframe = green!50!black, % A slightly darker green border
    boxsep = 5pt,
    arc = 4pt,
    outer arc = 4pt,
    boxrule = 0.5pt,
    left = 10pt,
    right = 10pt,
    top = 10pt,
    bottom = 10pt,
    % Add this line to ensure proper font encoding for Vietnamese characters
    fontupper = \setmainfont{Times New Roman}\normalfont,
}

% For math proofs or custom counters (tuỳ chọn nếu cần) - Can be removed if not used
\usepackage{lipsum}
\newcounter{mycounter}
\newcommand\showmycounter{\stepcounter{mycounter}\themycounter}
\newcommand\showlips{\stepcounter{mycounter}\lipsum[\value{mycounter}]}

% Others - Can be removed if not used
\usepackage{booktabs}
\usepackage{subcaption}
\usepackage{framed}
\usepackage{tikz}


%%%%%%%%%%%%%%%%%%%%%%%%%%%%%%%%%%%%%%%%%%%%%%%%%%%%%%%%%%%%%%%%%%%%%%%%%%%%%
%%%%%%%%%%%%%%%%%%%%%%%%%%%%%%%%%%%%%%%%%%%%%%%%%%%%%%%%%%%%%%%%%%%%%%%%%%%%%
%%%%%%%%%%%%%%%%%%%%%%%%%%%%%%%%%%%%%%%%%%%%%%%%%%%%%%%%%%%%%%%%%%%%%%%%%%%%%
\begin{document}

\maketitle % Call the redefined maketitle to display the custom title and date

\begin{summarybox}
    Buổi học số 2 (Thứ 3 + Thứ 4, 16/07/2025) Vì nội dung của buổi thứ 3 và 4 có liên kết và nội dung giống nhau nên mình ghép thành 1 phần với 8 nội dung chính:
    \begin{itemize}
        \item \textbf{Phần 1: Random Variable}
        \item \textbf{Phần 2: Random Discrete Variable}
        \item \textbf{Phần 3: Continuos Random Variable}
		\item \textbf{Phần 4: Expected Value, Variance, Standard Deviation và ứng dụng của chúng}
        \item \textbf{Phần 5: Mean, Median và ứng dụng của chúng}
        \item \textbf{Phần 7: Probability Distribution PMF, PDF và CDF}
        \item \textbf{Phần 8: Histogram và ứng dụng của nó}
        \item \textbf{Phần 9: Mở rộng: Histogram bằng numpy}
    \end{itemize}
\end{summarybox}

%? Giải thích tổng quan + công thức -> Ứng dụng
\newpage
\section{Phần 1: Random Variable}
\textbf{Tổng quan (giải thích định nghĩa và công thức)} \\
\textbf{Ví dụ cho Random Variable}

\section{Phần 2: Random Discrete Variable}
\textbf{Tổng quan (giải thích định nghĩa và công thức)} \\
\textbf{Ví dụ cho Random Discrete Variable} \\

Binomial Distribution (Phân phối nhị phân)


\section{Phần 3: Continuos Random Variable}
\textbf{Tổng quan (giải thích định nghĩa và công thức)}
\textbf{Ví dụ Continuos Random Variable}

\section{Phần 4: Probability Distribution PMF, PDF và CDF}
\textbf{Probability Distribution}

\textbf{From Discrete to Continuos Distribution}



\section{Phần 5: Expected Value, Variance, Standard Deviation và ứng dụng của chúng}
\subsection{Expected Value}

\subsection{Variance và ứng dụng của nó}

\subsection{Standard Deviation}


\section{Phần 6: Mean, Median và ứng dụng của chúng}
\subsection{Mean và ứng dụng của nó}

\subsection{Median và ứng dụng}

\section{Phần 7: Histogram và ứng dụng của nó}

\section{Phần 8: Mở rộng: Histogram bằng numpy}

\end{document}