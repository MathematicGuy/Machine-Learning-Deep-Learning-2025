\documentclass[11pt]{article}
% Font & ngôn ngữ tiếng Việt (pdfLaTeX)
\usepackage[utf8]{inputenc}
\usepackage[T5]{fontenc}

% Biblatex + biber
\usepackage[backend=biber, style=numeric, sorting=ynt]{biblatex}
\addbibresource{references.bib}
\DeclareBibliographyAlias{video}{misc}
\usepackage{hyperref}
\usepackage{hyperref}
\usepackage{listings}
\usepackage{xcolor}
\lstdefinestyle{shell}{
	language=bash,
	basicstyle=\ttfamily\small,
	commentstyle=\color{green!60!black},
	keywordstyle=\color{blue},
	stringstyle=\color{red},
	emphstyle=\color{purple},
	backgroundcolor=\color{gray!10},
	showstringspaces=false,
	frame=single,
	breaklines=true,
	emph={[1]\$\ },  % This is the key part: highlight the prompt
	emph=[1]{git, conda, m	kdir, cd, pip} % Also emphasize key commands
}

% Toán học & font Times
\usepackage{amsmath, amssymb, amsfonts, bm}

% Bảng biểu & căn lề
\usepackage{longtable}
\usepackage{array}
\usepackage{booktabs}

% Đồ họa & màu sắc
\usepackage{graphicx}
\usepackage{xcolor}
\usepackage{float}
\usepackage{subcaption}

% Liên kết & tham chiếu
\usepackage{hyperref}
\hypersetup{
    colorlinks=true,
    linkcolor=blue,
    urlcolor=red,
    pdftitle={Overleaf Example},
    pdfpagemode=FullScreen,
}
\usepackage{bookmark}

% Dấu tick và x
\usepackage{pifont}
\newcommand{\xmark}{\ding{55}}
\newcommand{\cmark}{\ding{51}}

% Tiêu đề tùy chỉnh
\usepackage{titling}
\setlength{\droptitle}{-10em}
\renewcommand{\maketitle}{%
    \begin{center}
        \fontsize{18}{20}\selectfont\textbf{Understand and Build CNN from the Grounth Up and Intuition}\\[1em]
        \fontsize{14}{16}\selectfont Nhóm AIO\_TimeSeries\\[0.5em]
        \fontsize{14}{16}\selectfont Ngày 18 tháng 11 năm 2025
    \end{center}
    \vspace{1.5em}
}

% Format section (không đánh số)
\usepackage{titlesec}
\titleformat{\section}{\normalfont\Large\bfseries}{}{0em}{}

% Code block
\usepackage{listings}
\definecolor{backcolour}{rgb}{0.95,0.95,0.92}
\lstset{
    backgroundcolor=\color{backcolour},
    basicstyle=\ttfamily\footnotesize,
    breaklines=true,
    numbers=left,
    numberstyle=\tiny\color{gray},
    captionpos=b
}

% Hộp màu
\usepackage[many]{tcolorbox}
\definecolor{lightgreenbox}{rgb}{0.85,0.95,0.85}
\newtcolorbox{summarybox}{
    colback=lightgreenbox,
    colframe=green!50!black,
    boxsep=5pt, arc=4pt,
    boxrule=0.5pt,
    left=10pt, right=10pt,
    top=10pt, bottom=10pt,
}


% Layout trang
\setlength{\topmargin}{-0.7in}
\setlength{\textheight}{9.25in}
\setlength{\oddsidemargin}{0in}
\setlength{\textwidth}{6.8in}

%%%%%%%%%%%%%%%%%%%%%%%%%%%%%%%%%%%%%%%%%%%%%%%%%%%%%%%%%%%%%%%%%%%%%%%%%%%%%
\begin{document}

\maketitle

% Note: Khi đăng Blog -> Làm đơn giản và trọng tâm nhất có thể.
\begin{summarybox}
Bài blog này tiếp cận CNN từ góc nhìn của Computer Vision và giải thích từ căn bản để người đọc có thể hiểu được ý nghĩa thực sự đằng sau CNN là gì. Blog sẽ bao quát các chủ đề sau: \\

\textbf{Phần 1: Vấn đề với MLP trong Phân loại Ảnh}
\textbf{Phần 2: Tổng quan kiến trúc CNN} \\ Từ Pixels đến Output qua các lớp Conv, ReLU, Pooling  \\ \\
\textbf{Phần 3: CNN Operation} \\ Convolution, Padding, Stride, ReLU trong CNN  \\ \\
\textbf{Phần 4: CNN Backpropagation} \\ Cơ chế lan truyền ngược trong mạng tích chập  \\ \\
\textbf{Phần 5: Advance CNN} \\ Pooling, Batch Normalization, 1x1 Convolution \\ \\
\textbf{Phần 6: Thực hành Code CNN dưới góc nhìn  lập trình viên} \\ Triển khai CNN bằng tư duy lập trình hướng đối tượng OOP
\end{summarybox}

% --- Blog Outline ---
\section{Phần 1: Giới thiệu về Computer Vision \& Tensor}
Trong phần này, chúng ta tìm hiểu cách hình ảnh được biểu diễn dưới dạng các Tensor 1D, 2D và 3D.
\cite{stanford_cs231n}
\textbf{Vấn đề của MLP (Multi-Layer Perceptron):}
Khi xử lý hình ảnh lớn, số lượng node đầu vào tăng vọt khiến MLP chạy rất chậm và dễ bị quá tải tham số. Giải pháp là sử dụng lớp Convolution để đơn giản hóa các đặc trưng (features) trước khi đưa vào MLP. CNN không thay thế MLP mà đóng vai trò là bộ trích xuất đặc trưng (feature extractor) mạnh mẽ hơn.

\section{Phần 2: Tổng quan kiến trúc CNN}
Kiến trúc CNN bao gồm các thành phần chính giúp xử lý thông tin không gian hiệu quả.
Quy trình tính toán tổng quát:
\begin{quote}
(Pixels + bias) $\rightarrow$ (Conv $\rightarrow$ ReLU) $\times 2$ $\rightarrow$ Pooling $\rightarrow$ (Flatten + bias) $\rightarrow$ Softmax $\rightarrow$ Output
\end{quote}
Sử dụng công cụ trực quan hóa như CNN Explainer để thấy sự khác biệt trước và sau khi qua các lớp.

\section{Phần 3: Đi sâu vào các phép toán CNN}
Dựa trên tài liệu của Stanford CS231n:
\begin{itemize}
    \item \textbf{Convolution:} Phép tính tích chập lặp lại để trích xuất đặc trưng cục bộ.
    \item \textbf{Padding:} Khi nào cần thiết để giữ lại kích thước của matrix?
    \item \textbf{Stride:} Bước nhảy của kernel. Stride = 1 nghĩa là kernel di chuyển từng pixel. Ảnh hưởng của Stride đến việc trích xuất đặc trưng cục bộ (local) và toàn cục (global).
    \item \textbf{ReLU:} Đưa tính phi tuyến vào mô hình và loại bỏ nhiễu từ các giá trị âm.
\end{itemize}

\section{Phần 4: CNN Backpropagation}
Giải thích cách cập nhật trọng số cho các bộ lọc (filters) thông qua đạo hàm của phép toán tích chập.

\section{Phần 5: CNN Nâng cao \& Lập trình OOP}
Tìm hiểu về Pooling (Max/Average), Batch Normalization và đặc biệt là ý nghĩa của phép tích chập $1 \times 1$. Cuối cùng là triển khai code thực tế bằng Python theo hướng đối tượng (OOP).

\printbibliography % Hiển thị danh sách tham khảo



\end{document}
